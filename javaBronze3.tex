\documentclass[12pt]{jarticle}

\begin{document}

次のコードをコンパイル、実行した時の結果として、正しいものを選びなさい。


何も表示されない。
「56789」と1回表示される
「56789」と5回表示される
コンパイルエラーになる。
実行時に例外が発生する。

次のSampleクラスを継承したサブクラスを定義するときに、
サブクラスに定義したメソッドのうち、Sampleクラスのメソッドを正しくオーバーライドしているものを選びなさい。


public void  methodA(){}
public  void methodB( long a ){}
public int methodC( char a, int b){return 0;}
public int methodD( int i ){return 1;}

次のコードをコンパイル、実行した時の結果として、正しいものを選びなさい。(1つ選択)

「none aline」と表示される
「null aline」と表示される
「aline aline」と表示される
コンパイルエラーになる
実行時に例外が発生する

ある企業は、GUIベースのアプリケーション開発を希望しており、将来的な拡張として、
Webベースのアプリケーションへの移行を予定している。
このアプリケーションを作成するには、どのJavaテクノロジを使用するのが良いか。
正しいものを選びなさい。(1つ選択)

Java SE
Java ME
Java EE
Java DB

次のプログラムをコンパイル、実行した時の結果として、
正しいものを選びなさい。(1つ選択)

public interface Sample {
    void test();
}

public class A implements Sample{
    public void test(){
        System.out.println( "A");
    }
}

public class B extends A {
    public void test(){
        System.out.println( "B");
    }
}

public class Main{
    public  static void main( String[] args){
        Sample[] samples = { new A(), new B()};
        for ( Sample s : samples){
            a.test();
        }
    }
}


「A」「B」の順に表示される
「B」「A」の順に表示される
「A」「A」の順に表示される
「B」「B」の順に表示される
Bクラスでコンパイルエラーが発生する
Mainクラスでコンパイルエラーが発生する

次のコードをコンパイル、実行した時の結果として、正しいものを選びなさい。(1つ選択))

public  class P249 {
    public  static void main( String[] args){
        int i = 5;
        System.out.println((i += 5 ) + ":" + ( i--));
    }
}

「5:5」と表示される
「5:4」と1回表示される
「5:9」と表示される
「10:10」と表示される
「10:9」と表示される

以下の中から、アクセス修飾子privateで修飾
できるものを選びなさい。
(3つ選択)

インタフェースのフィールド
クラスのコンストラクタ
クラスのフィールド
クラスの抽象メソッド
インタフェースのメソッド
クラスの具象メソッド

次のコードのコンパイルを成功させ、実行結果が「Refresh L」となるようにしたい。
空欄にあてはまるコードを選びなさい。
(1つ選択)


%% p250.java


%%p251

次のプログラムをコンパイル、実行した時の結果として、正しいものを選びなさい。
(1つ選択)

%%p251.java

「1:2:3」と表示される
「0:2:3」と表示される
コンパイルエラーになる
実行時に例外がスローされる

クラスの宣言として、有効なものを選びなさい。(3つ選択)
public class Test extends java.lang.* { }
public class Test extends java.lang.Object{}
final class Test{}
public class Test{}
public class Test implements Object{}

次のコードをコンパイル、実行した時の結果として、正しいものを選びなさい。
(1つ選択)

%% P252.java

「6:14」と表示される
「6:20」と表示される
「12:20」と表示される
コンパイルエラーとなる

%% 12

次のコードをコンパイル、実行した時の結果として、正しいものを選びなさい。(1つ選択)

「123」と1回表示される
「0123」と1回表示される
「1234」と1回表示される
「1234」が無限に表示される
「1234」が無限に表示される
コンパイルエラーとなる

%% Q.13

次のコードをコンパイル、実行した時の結果として、正しいものを選びなさい。(1つ選択)

%% P2532.java

何も表示されない
「01234」と表示される
「012345と表示される
「1234と表示される
「12345と表示される
コンパイルエラーになる

%% Q.14

次のプログラムの6行目「// inser code here 」に入るコードとして、正しいものを選びなさい。
(1つ選択)

%% P254.java

this.price = 100 ;
this ( "sample");

this( "sample");
this.price = 100;

this( TMP);
this.price  100;

this.price = 100;
this( TMP);

Sample( "sample");
this.price = 100;

%% Q.15

次のコードをコンパイル、実行した時の結果として、正しいものを選択しなさい。
(1つ選択)

%% 


\end{document}
